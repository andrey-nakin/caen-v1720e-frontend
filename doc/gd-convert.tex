\section{Введение}

\MIDAS{} использует программу \APP{mlogger} для записи результатов измерений в файлы. Файлы имеют расширение \FILE{.mid}, если не используется сжатие, или же другое, например, \FILE{.mid.gz}, если в настройках \APP{mlogger} указано сжатие в формат GZip. Подробную информацию о \APP{mlogger} можно узнать в официальной документации \cite{MidasWikiMLogger}.

MID-файлы имеют бинарный формат. Для их конвертации в другие форматы используется программа \GDCONVERT{}.

\section{Параметры командной строки}

\GDCONVERT{} является приложением командной строки. Формат команды:

\begin{lstlisting}[language=bash]
gd-convert -f<format> [-F<file name>] \
  <MID file 1> [ <MID file 2> ... ]
\end{lstlisting}

На вход программы подаётся одно или несколько имён MID-файлов. MID-файлы могут быть в сжатом виде, \GDCONVERT{} самостоятельно произведёт распаковку. Вместо точного имени MID-файла можно указать т.~н. \TERM{маску}, в которой часть символов заменены на специальные символы \FILE{?} и \FILE{*}, где:

\begin{itemize}
\item \FILE{?} обозначает единичный произвольный символ в имени файла;
\item \FILE{*} обозначает ноль, один или больше произвольных символов в имени файла.
\end{itemize}

Например, маска \FILE{run0000*.mid} обозначает все файлы с именами \FILE{run00000.mid}, \FILE{run00001.mid} и т.~д. Подробнее о масках файлах в ОС~Linux можно ознакомиться, например, в \cite{LinuxFileMask}.

Обязательный параметр \CMDARG{-f} определяет результирующий формат. В настоящей версии поддерживаются следующие форматы:

\begin{itemize}
\item \CMDARG{simple}~--- простой текстовый формат, позволяющий производить разбор данных произвольными инструментами.
\item \CMDARG{binary}~--- бинарный формат, соответствующий формату ранее написанного ПО для обработки данных от 1~ГГц дигитайзера.
\end{itemize}

Необязательный параметр \CMDARG{-F} определяет имя файла, в который записаются данные в нужном формате. Если параметр не указан, \GDCONVERT{} создаёт в текущей директории новый файл для каждого нового MID-файла. Если в качестве имени файла указано \CMDARG{stdout}, вывод данных производится не в файл, а в стандарный поток вывода \cite{LinuxInOutStream}, что позволяет передавать результат работы \GDCONVERT{} непосредственно в другую программу, без использования файловой системы.

\subsection{Примеры использования}

Пример 1: конвертация 2 файлов в текстовый формат. В текущей директории будут созданы файлы \FILE{run00001.txt} и \FILE{run00002.txt}.

\begin{lstlisting}[language=bash]
gd-convert -fsimple run00001.mid.gz run00002.mid.gz
\end{lstlisting}

\bigskip

Пример 2: конвертация нескольких файлов, удовлетворяюших заданной маске, в бинарный формат. В текущей директории будут созданы файлы \FILE{run00001.data}, \FILE{run00002.data} и~т.~д.

\begin{lstlisting}[language=bash]
gd-convert -fbinary run0000*.mid.gz
\end{lstlisting}

\bigskip

Пример 3: конвертация 2 файлов в текстовый формат. В текущей директории будет создан единственный файл \FILE{res.txt} с результатами из обоих MID-файлов.

\begin{lstlisting}[language=bash]
gd-convert -fsimple -Fres.txt run00001.mid.gz run00002.mid.gz
\end{lstlisting}

\bigskip

Пример 4: конвертация 2 файлов в текстовый формат. Результат конвертации записывается в стандартный поток вывода.

\begin{lstlisting}[language=bash]
gd-convert -fsimple -Fstdout run00001.mid.gz run00002.mid.gz
\end{lstlisting}
