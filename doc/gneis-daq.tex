\documentclass[12pt, a4paper, oneside, onecolumn]{book}
\usepackage[utf8]{inputenc}
\usepackage[russian]{babel}
\usepackage{cmap}
\usepackage{gensymb}
\usepackage{graphicx}
\usepackage{indentfirst}
\usepackage{hyperref}
\usepackage{listings}
\usepackage[backend=biber, style=gost-numeric]{biblatex} 
\addbibresource{gneis-daq.bib}

\lstset{basicstyle=\footnotesize\ttfamily,breaklines=false}
\lstset{frame=tb}

\newcommand{\GD}{{\tt gneis-daq}}
\newcommand{\MIDAS}{\mbox{MIDAS}}
\newcommand{\ROOT}{\mbox{ROOT}}
\newcommand{\ROOTJS}{\mbox{JSROOT}}
\newcommand{\ROOTANA}{\mbox{ROOTANA}}
\newcommand{\APP}[1]{\mbox{\tt #1}}
\newcommand{\TERM}[1]{{\it #1}}
\newcommand{\BANK}[1]{{\tt #1}}
\newcommand{\ODBNODE}[1]{<<{\tt #1}>>}
\newcommand{\CPPCLASS}[1]{\mbox{\tt #1}}
\newcommand{\DIRECTORY}[1]{<<{\tt #1}>>}
\newcommand{\FILE}[1]{<<{\tt #1}>>}
\newcommand{\COMMAND}[1]{{\tt #1}}
\newcommand{\PACKAGE}[1]{{\tt #1}}
\newcommand{\ENV}[1]{{\tt #1}}
\newcommand{\BITVALUE}[1]{<<#1>>}
\newcommand{\NOTE}{{\bf Внимание:}}

\input{version.tex}

\title{\GD{}: набор программ по сбору и анализу данных для установки ГНЕЙС. \\
Версия: \GDVER
}

\author{Накин~А.~В.}
\date{Ревизия документа: 0.1\\Февраль 2019}

\begin{document}

\maketitle

\tableofcontents

\chapter{Введение}

\GD{} --- набор программ по сбору данных, созданный для использования на установке ГНЕЙС \cite{shcherb2018}, но который может быть частично или полностью  использоваться и в других экспериментах с аналогичной аппаратной базой.

\GD{} предназначен для считывания и записи экспериментальных данных в числовой форме и их оперативного анализа, в том числе удалённого.

\GD{} разработан на базе системы сбора данных \MIDAS{} \cite{midas} с использованием программного пакета ROOT \cite{RootHome} и других.

\section{Системные требования}

\subsection{Операционная система}

Как сам \GD{}, так и все используемые программные компоненты написаны на кросс-платформенных языках программирования, таких как С и C++. Это теоретически позвляет использовать \GD{} в различных операционных системах.

Практически же, разработка и тестирование \GD{} производились только в различных версиях ОС Linux, таких как Debian и openSUSE. Использование \GD{} в других ОС, например в MS Windows, может потребовать некоторой доработки.

\section{Общие сведения о \MIDAS{}}

\MIDAS{} (Maximum Integrated Data Acquisition System) --- современная система сбора данных для физических экспериментов с объёмом данных масштаба малого и среднего, вплоть до десятков Гб/сек. Это совместная разработка PSI \cite{psi} и TRIUMF \cite{triumf}, первая версия выпущена в 1997 году.

\MIDAS{} представляет собой библиотеку функций и коллекцию программ на её основе, в их число входят:

\begin{itemize}
\item \APP{odbedit}: позволяет редактировать базу данных эксперимента; также предоставляет команды для начала, остановки или приостановки записи данных.
\item \APP{mlogger}: регистратор, записывает результаты эксперимента в числовой форме на диск или магнитную ленту. Для экономии места данные могут сжиматься <<на лету>>.
\item \APP{mhttpd}: позволяет удалённо редактировать базу данных, управлять ходом эксперимента и компонентами системы посредством веб-интерфейса.
\item и множество других.
\end{itemize}

\subsection{База данных эксперимента}

Информация об эксперименте, такая как настройки оборудования, режимы работы каждого из компонентов, протокол использования оборудования и пр. содержится в специальной базе данных --- \TERM{Online Data Base} (ODB) \cite{MidasWikiODB}.

База данных ODB --- нереляционная и имеет древовидную структуру, где каждый раздел может иметь один или несколько дочерних разделов (или не иметь таковых), а также записи, которые собственно и хранят информацию. Разделы и записи идентифицируются по именам. ODB является, так сказать, аналогом файловой системы, где вместо директорий --- разделы, а вместо файлов --- записи.

Записи могут быть числового или строкового типа с фиксированной максимальной длиной. Запись может представлять собой массив значений одинакового типа.

Как правило, каждый компонент \MIDAS{} имеет свою собственный раздел в ODB, где хранит свои настройки. Например, для настроек оборудования имеется стандартный  раздел \ODBNODE{/Equipment} \cite{MidasWikiEquipment}, устройство с названием \ODBNODE{v1720} хранит свои специфичные настройки в дочернем разделе \ODBNODE{/Equipment/v1720}, а внутри этого раздела запись \ODBNODE{link\_num} содержит значение параметра <<Link Number>>.

Просматривать и редактировать содержимое ODB можно при помощи утилиты командной строки \APP{odbedit}. Но удобнее это делать в веб-интерфейсе, предоставляемом веб-сервером \APP{mhttpd} \cite{MidasWikiMhttpd}.

Каждый новый эксперимент может иметь как свою собственную базу данных, так и пользоваться общей, в зависимости от требований данного эксперимента.

\subsection{Фронтенды}
\label{sec-midas-frontend}

Связь с оборудованием осуществляется посредством небольших программ, называемых \TERM{фронтендами} (\TERM{Frontend Application}) \cite{MidasWikiFrontend}. Фронтенд управляет устройством, производит его конфигурирование и обеспечивает приём данных, идущих от устройства. Как правило, каждое устойство, задействованное в сборе данных, имеет свой собственный фронтенд.

\MIDAS{} имеет в своём составе набор фронтендов для типового оборудования физического эксперимента. Но, как правило, наличие нестандартного оборудования требует от экспериментатора написания специальной программы-фронтенда.

\subsection{Анализаторы}
\label{sec-midas-analyzer}

\MIDAS{} предоставляет возможность оперативного контроля за экспериментом. Для этого в систему добавляются небольшие программы-\TERM{анализаторы} (\TERM{Analyzers}). Анализатор <<прослушивает>> поток данных, идущий от фронтендов, и производит обработку этих данных, например накапливает статистику по определённым событиям. Далее, обработанная информация может сохраняться на диске или оперативно отображаться на экране персонального компьютера (ПК), например в виде диаграмм.

Система не ограничивает количество анализаторов, способы анализа и визуалиации. Но экспериментаторы должны позаботиться о написании анализаторов, соответствующих их нуждам.

\subsection{События}
\label{sec-midas-event}

В системе \MIDAS{} единицей хранения и передачи информации является \TERM{событие} (\TERM{Event}), которое в свою очередь состоит из одного или нескольких \TERM{банков} (\TERM{Banks}) \cite{MidasWikiEvent}. Каждый банк имеет 4-х буквенное название, по которому определяется его тип. Событие может включать в себя банки разного типа.

События формируются программами-фронтендами на основе данных, полученных от измерительного оборудования. Сформированные события передаются в систему. Далее, эти события могут быть:

\begin{itemize}
\item обработаны <<на лету>> при помощи программ-анализаторов;
\item записаны на диск программой-регистратором для последующего анализа.
\end{itemize}

\subsection{Прочие характеристики}

\MIDAS{} --- распределённая система, то есть её компоненты могут работать на разных рабочих станциях, связанных в общую компьютерную сеть (локальную или глобальную, такую как интернет). Например, наблюдать накопление статистики в реальном времени можно на ПК, который территориально удалён от экспериментальной площадки, что увеличивает удобство экспериментатора, уменьшает дозовую нагрузку.

\MIDAS{} --- расширяемая система, позволяющая добавлять новые компоненты. Эта работа значительно упрощается благодаря общей библиотеке, которая берёт на себя такие задачи, как доступ к базе данных, связь компонентов между собой, протоколирование событий и другие.

Основные компоненты \MIDAS{} написаны на C/C++. Согласно документации, поддерживаются операционные системы Linux, MacOS, и MS Windows.

\subsection{Где используется}

На базе \MIDAS{} реализованы эксперименты, ведущиеся в PSI, TRIUMF, CERN, Fermilab и др. С их неполным перечнем можно познакомиться в \cite{MidasWikiWorld}.

\subsection{Условия использования}

\MIDAS{} распространяется в исходных кодах под лицензией GNU (General Public License) \cite{midas}.

\section{Общие сведения о \ROOT{}}

\ROOT{} \cite{RootHome} --- пакет программ и библиотек для обработки экспериментальных данных, преимущественно в области физики высоких энергий, но также и в других областях, где требуется статистическая обработка данных и их визуализация.

Разработан и поддерживается в CERN начиная с 1994 года, современная версия, написанная на C++, разрабатывается с 2003 года.

\ROOT{} --- крупный пакет с богатой функциональностью, но в рамках \GD{} используются его следующие возможности:

\begin{itemize}
\item Построение графиков с изображением wave-форм и других функций.
\item Построение гистограмм с функциями распределения.
\item Предоставление удалённого доступа к построенным графикам и гистограммам через веб-интерфейс (посредством класса \CPPCLASS{THttpServer}). Данные возвращаются в формате JSON и далее используются на стороне браузера для отрисовки.
\item Визуализация графиков и гистограм в виде интерактивных диаграмм в браузере (посредством библиотеки \ROOTJS{}).
\end{itemize}

\MIDAS{} сам по себе не нуждается в \ROOT{}, но его могут использовать программы-анализаторы, используемые в эксперименте (стр.~\pageref{sec-midas-analyzer}). Кроме того, \ROOT{} необходим для использования \ROOTANA{} (стр.~\pageref{sec-rootana}).

\subsection{Условия использования}

\ROOT{} распространяется в исходных кодах под лицензией LGPL, отдельные его компоненты --- под лицензией GPL \cite{RootLicense}.

\section{Общие сведения о \ROOTJS{}}

Поле того, как \ROOT{} сформировал данные для графиков и гистограмм, их необходимо каким-то образом визуализировать. \ROOT{} умеет самостоятельно отрисовывать диаграмм на экране ПК. Но зачастую задача по сбору статистики и построению диаграмм может быть отделена от задачи по их отрисовке. Примеры:

\begin{itemize}

\item Диаграммы могут быть построены в процессе измерений и сохранены в виде файлов. Но их визуализация и анализ производятся позднее по необходимости.

\item Диаграммы формируются при помощи программ-анализаторов (стр.~\pageref{sec-midas-analyzer}) на компьютере, подключённом к экспериментальному оборудованию. Но наблюдать эти диаграммы может быть удобнее с удалённого рабочего места, связанного с экспериментальным компьютером по сети. Заниматься же отрисовкой непосредственно на экспериментальном компьютере --- значит отбирать его вычислительные ресурсы (то есть увеличивать расход процессорного времени и оперативной памяти).

\end{itemize}

С другой стороны, развитие веб-технологий в настоящее время позволяет рисовать сложные диаграммы непосредственно в браузере. Причём эти диаграммы могут быть интерактивными, т.~е. реагирующими на команды пользователя.

Вероятно, изложенное выше и послужило отправной точкой для написания библиотеки JavaScript-функций \ROOTJS{} \cite{JsRootHome}.

\ROOTJS{} принимает на вход данные графиков, гистограмм и пр., сформированные посредством \ROOT{}, и рисует на их основе диаграммы в окне браузера. 

Диаграммы являются интерактивными, в частности пользователь может:

\begin{itemize}
\item назначить или отменить логарифмический масштаб по одной или нескольким осям;
\item увеличить масштаб для детального отображения интересующей области.
\end{itemize}

Разработка \ROOTJS{} ведётся с 2014 года силами разработчиков CERN.

\subsection{Условия использования}

\ROOTJS{} распространяется в исходных кодах под лицензией MIT \cite{JsRootHome}.

\section{Общие сведения о \ROOTANA{}}
\label{sec-rootana}

В разделе~\ref{sec-midas-analyzer} (стр.~\pageref{sec-midas-analyzer}) говорилось о программах-анализаторах, предназначенных для анализа экспериментальных данных. \MIDAS{} предоставляет библиотеку  функций на языке C для написания таких программ.

Но зачастую удобнее иметь библиотеку на C++, как языке программирования с гораздо более богатыми возможностями. Также, часто работа анализаторов сводится к формирования диаграмм и гистограмм, а значит имеет смысл использовать пакет \ROOT{}.

Тут на помощь приходит небольшая библиотека на языке C++, названная \ROOTANA{} (от \ROOT{} и ANAlyzer) \cite{MidasWikiRootana}. Эта библиотека предлагает следующие возможности:

\begin{itemize}
\item операции с файлами результатов, включая сжатые файлы;
\item операции с файлами в формате {\tt root}, в которых хранятся диаграммы и гистограммы;
\item операции с базой данных ODB;
\item доступ к потоку событий, идущих от фронтендов;
\item прочие вспомогательные классы для работы с \ROOT{} и \MIDAS{}.
\end{itemize}

Анализатор, написанный на основе \ROOTANA{}, может использоваться не только для оперативного анализа {\it в процессе} эксперимента: он также может использоваться для {\it пост-фактум} анализа, когда данные берутся из ранее созданных файлов с результатами.

\subsection{Условия использования}

\ROOTANA{} распространяется в исходных кодах \cite{RootanaHome}. Информация о лицензии и условиях использования не обнаружена на сайте.

\chapter{Установка программного обеспечения}

Данная глава освещает процесс установки \GD{} и сопутствующего программного обеспечения (ПО) в операционной системе Linux. Описание работы в других ОС выходит за рамки данного документа. В случае проблем рекомендуем обратиться к документации по каждому конкретному компоненту ПО.

\section{Общие сведения и соглашения}

Информация, изложенная в этом разделе, требует некоторых минимальных знаний по работе в ОС Linux, в частности:

\begin{itemize}
\item назначение команды \APP{sudo};
\item назначение переменных окружения (\TERM{environment variables});
\item назначение символических ссылок (\TERM{symlinks});
\item что такое пакеты программного обеспечения (\TERM{packages}) и как их устанавливать;
\item простейшие манипуляции с файловой системой, такие как переход по каталогам, создание каталогов и пр;
\item манипуляции с архивами в формате {\tt .tar.gz}.
\end{itemize}

\subsection{Поддержка нескольких пользователей}
\label{sec_multiuser}

Перед началом установки ПО необходимо решить, будет ли экспериментальный ПК использоваться исключительно экспериментатором, или же на нём предполагается работа и других пользователей, например, разработчиков ПО или участников других экспериментов.

Если ПК используется несколькими пользователями, то ПО рекомендуется устанавливать в общие системные каталоги, чтобы все пользователи имели к нему доступ. В ОС Linux программные компоненты, не являющиеся частью системы и не входящие в состав репозиториев, как правило, устанавливаются в каталог \DIRECTORY{/opt}.

Если же ПК используется исключительно в целях одного эксперимента, или же если не предполагается использование компонентов ПО другими пользователями, то рекомендуется следовать соглашениям, принятым в \MIDAS{}, и производить установку в каталог \DIRECTORY{\~{}/packages}. Здесь и далее символ \DIRECTORY{\~{}} обозначает домашний каталог пользователя.

\section{Установка стандартных пакетов}

Перед началом установки ПО необходимо установить необходимые пакеты из стандарных репозиториев ОС Linux.

\subsection{Установка пакетов в ОС семейства openSUSE}

Установите пакеты следующей командой:

\begin{lstlisting}[language=bash]
sudo zypper install git bash cmake gcc-c++ gcc binutils \
  openssl-devel libxml2-devel kernel-source
\end{lstlisting}

Для сборки документации, входящей в состав \GD{}, необходимо установить следующие пакеты:

\begin{lstlisting}[language=bash]
sudo zypper install texlive texlive-babel-bulgarian \
  texlive-babel-bulgarian-doc texlive-babel-serbian \
  texlive-babel-serbianc texlive-babel-serbianc-doc \
  texlive-babel-serbian-doc texlive-babel-ukraineb \
  texlive-babel-ukraineb-doc texlive-cbfonts \
  texlive-cbfonts-doc texlive-cbfonts-fonts texlive-cmcyr \
  texlive-cmcyr-doc texlive-cmcyr-fonts \
  texlive-collection-langcyrillic texlive-cyrplain \
  texlive-diagnose texlive-diagnose-doc texlive-disser \
  texlive-disser-doc texlive-eskd texlive-eskd-doc \
  texlive-eskdx texlive-eskdx-doc texlive-gost \
  texlive-gost-doc texlive-hyphen-bulgarian \
  texlive-hyphen-mongolian texlive-hyphen-serbian \
  texlive-hyphen-ukrainian texlive-lcyw texlive-lcyw-doc \
  texlive-lh texlive-lhcyr texlive-lh-doc \
  texlive-lshort-bulgarian texlive-lshort-mongol \
  texlive-lshort-ukr texlive-mongolian-babel \
  texlive-mongolian-babel-doc texlive-montex \
  texlive-montex-doc texlive-montex-fonts texlive-mpman-ru \
  texlive-pst-eucl-translation-bg texlive-serbian-apostrophe \
  texlive-serbian-apostrophe-doc texlive-serbian-date-lat \
  texlive-serbian-date-lat-doc texlive-serbian-def-cyr \
  texlive-serbian-def-cyr-doc texlive-serbian-\
  lig texlive-serbian-lig-doc texlive-texlive-ru \
  texlive-texlive-sr
\end{lstlisting}

\section{Установка \ROOT{}}

Данный раздел --- упрощённое изложение официальной инструкции по сборке \ROOT{} \cite{RootBuild}. Оно не освещает все нюансы и возможные проблемы, с которыми можно столкнуться в процессе установки. В этом случае рекомендуем обратиться к официальной инструкции.

Перед установкой \ROOT{} необходимо убедиться, что все необходимые библиотеки установлены. Их  список приведён в \cite{RootPrerequisites}, там же приведены установочные команды для основных операционных систем.

\subsection{Выбор версии \ROOT{}}

Список версий \ROOT{}, доступных для скачивания, находится в \cite{RootDownload}. Рекомендуется всегда использовать последнюю  production-версию, отмеченную аббревиатурой <<Pro>>.

Если версия компилятора C++, имеющаяся в наличии на экспериментальном ПК, является устаревшей и не позволяет сборку последней версии \ROOT{}, то допускается использование более ранних версий с номером не менее 6.14.

\ROOT{} распространяется как в исходных кодах, так и в виде бинарных сборок. Первый вариант предпочтительнее, поскольку это гарантирует правильность работы \ROOT{} в данной версии ОС. Далее рассмотрим оба варианта установки.

\subsection{Установка \ROOT{} из исходных кодов}
\label{sec-build-root-from-src}

\begin{enumerate}

\item Выберите нужную версию \ROOT{} в \cite{RootDownload} и найдите архив с исходными кодами в разделе <<Source distribution>>  . Пусть, для примера, файл архива имеет следующее имя:

\begin{lstlisting}[language=bash]
root_v6.14.06.source.tar.gz
\end{lstlisting}
 
\noindent где <<6.14.06>>~--- версия \ROOT{}.

\item Скачайте файл архива во временный каталог \DIRECTORY{/tmp}.

\item Распакуйте архив:

\begin{lstlisting}[language=bash]
cd /tmp
tar -xzf root_v6.14.06.source.tar.gz
\end{lstlisting}

\item После распаковки в каталоге \DIRECTORY{/tmp} должен появиться подкаталог \DIRECTORY{root-6.14.06}. Перейдите в него:

\begin{lstlisting}[language=bash]
cd root-6.14.06
\end{lstlisting}

\item Создайте в каталоге \DIRECTORY{/tmp/root-6.14.06} подкаталог \DIRECTORY{obj} и перейдите в него:

\begin{lstlisting}[language=bash]
mkdir obj
cd obj
\end{lstlisting}

\item 
\label{item-select-root-dir}

Если \ROOT{} будет использоваться несколькими пользователями ПК (стр.~\pageref{sec_multiuser}), установим его в каталог \DIRECTORY{/opt/root-6.14.06}. Для этого выполните  следующую команду:

\begin{lstlisting}[language=bash]
cmake -Dxml=ON -Dhttp=ON \
  -DCMAKE_INSTALL_PREFIX=/opt/root-6.14.06 ..
\end{lstlisting}

Если же использование \ROOT{} другими пользователями не предполагается, выполните следующую команду:

\begin{lstlisting}[language=bash]
cmake -Dxml=ON -Dhttp=ON \
  -DCMAKE_INSTALL_PREFIX=~/packages/root-6.14.06 ..
\end{lstlisting}

\item Если все необходимые библиотеки установлены и версия компилятора C++ достаточно <<свежая>>, команда завершится следующим сообщением в консоли:

\begin{lstlisting}
-- Generating done
-- Build files have been written to: /tmp/root-6.14.06/obj
\end{lstlisting}

В противном случае необходимо ознакомиться с сообщениями в консоли или просмотреть файлы протокола \APP{cmake}, чтобы устранить ошибку.

\item Подготовка к сборке завершена. Если \ROOT{} устанавливается в каталог \DIRECTORY{/opt}, то сборка и установка выполняется следующей командой:

\begin{lstlisting}[language=bash]
sudo cmake --build . --target install
\end{lstlisting}

При этом у пользователя может быть запрошен пароль администратора системы.

Если же установка \ROOT{} производится в каталог \DIRECTORY{\~{}/packages}, то используйте следующую команду:

\begin{lstlisting}[language=bash]
cmake --build . --target install
\end{lstlisting}
 
Сборка может занять порядка часа или больше, в зависимости от производительности компьютера.

\item Удалите временный каталог \DIRECTORY{/tmp/root-6.14.06} для освобождения места на диске:

\begin{lstlisting}[language=bash]
cd /tmp
rm -fr root-6.14.06
\end{lstlisting}

\item Создайте символическую ссылку на каталог с установленным \ROOT{}:
\label{item-root-symlink}

\begin{lstlisting}[language=bash]
cd /opt
sudo ln -s root-6.14.06 root
\end{lstlisting}

Если \ROOT{} установлен в директорию \DIRECTORY{\~{}/packages}, используйте следующие команды:

\begin{lstlisting}[language=bash]
cd ~/packages
ln -s root-6.14.06 root
\end{lstlisting}

Таким образом, символическая ссылка используется как псевдоним. Это позволяет одновременно иметь несколько версий \ROOT{} и легко переключаться между ними, перенаправляя символическую ссылку на нужный каталог.

\end{enumerate}

\subsection{Установка \ROOT{} из бинарной сборки}

\begin{enumerate}

\item Выберите нужную версию \ROOT{} в \cite{RootDownload} и найдите архив с бинарной сборкой, который соответствует Вашей ОС, в разделе <<Binary distributions>>. Пусть, для примера, у Вас установлена Linux CentOS~7, и файл архива имеет следующее имя:

\begin{lstlisting}[language=bash]
root_v6.14.06.Linux-centos7-x86_64-gcc4.8.tar.gz
\end{lstlisting}
 
\noindent где <<6.14.06>>~--- версия \ROOT{}.

\item Скачайте архив и распакуйте его в нужную директорию, в зависимости от того, будет ли \ROOT{} использоваться одними или несколькими пользователями (см. пункт~\ref{item-select-root-dir} раздела~\ref{sec-build-root-from-src} на стр.~\pageref{item-select-root-dir}).

\item Создайте символическую ссылку (см. пункт~\ref{item-root-symlink} раздела~\ref{sec-build-root-from-src} на стр.~\pageref{item-root-symlink}).
\end{enumerate}

\subsection{Настройка окружения \ROOT{}}

После того, как \ROOT{} успешно установлен, добавьте следующие строки в файл \FILE{\~{}/.profile}:

\begin{lstlisting}[language=bash]
export ROOTSYS=/opt/root
export PATH=$ROOTSYS/bin:$PATH
export LD_LIBRARY_PATH=$ROOTSYS/lib:$LD_LIBRARY_PATH
export PYTHONPATH=$ROOTSYS/lib:$PYTHONPATH
export CPATH=$ROOTSYS/include:$CPATH
export C_INCLUDE_PATH=$ROOTSYS/include:$C_INCLUDE_PATH
export CPLUS_INCLUDE_PATH=$ROOTSYS/include:$CPLUS_INCLUDE_PATH
\end{lstlisting}

Если \ROOT{} установлен в директорию \DIRECTORY{\~{}/packages}, то скорректируйте первую строку соответствующим образом.

Далее, добавьте следующие строки в файл \FILE{\~{}/.bashrc}:

\begin{lstlisting}[language=bash]
source $ROOTSYS/bin/thisroot.sh
\end{lstlisting}

Для того, чтобы изменения вступили в силу, завершите сеанс работы в ОС Linux и начните новый.

\section{Установка \ROOTJS{}}

Если предполагается работа в многопользовательском режиме (стр.~\pageref{sec_multiuser}), то рекомендуется установить \ROOTJS{} в каталог \DIRECTORY{/opt/jsroot}, в противном случае рекомендуемый каталог~--- \DIRECTORY{\~{}/packages/jsroot}. В дальнейших примерах предполагается первый вариант установки. Если же \ROOTJS{} устанавливается в домашний каталог, то скорректируйте названия директорий соответствующим образом и опустите команду \APP{sudo}.

\begin{enumerate}
\item Перейдите в каталог \DIRECTORY{/opt}:

\begin{lstlisting}[language=bash]
cd /opt
\end{lstlisting}

\item Скачайте репозиторий с исходными кодами \ROOTJS{}:

\begin{lstlisting}[language=bash]
sudo git clone https://github.com/root-project/jsroot.git
\end{lstlisting}

\item Перейдите в основную ветвь репозитория:

\begin{lstlisting}[language=bash]
cd jsroot
sudo git checkout master
\end{lstlisting}

\item Настройте переменные окружения. Для этого добавьте следующие строки в файл \FILE{\~{}/.profile}:

\begin{lstlisting}[language=bash]
export JSROOTSYS=/opt/jsroot
\end{lstlisting}

\end{enumerate}

Для того, чтобы изменения вступили в силу, завершите сеанс работы в ОС Linux и начните новый.

Если \ROOTJS{} был установлен ранее и необходимо обновить его версию, то выполните следующие команды:

\begin{enumerate}
\item Перейдите в каталог \DIRECTORY{/opt/jsroot}:

\begin{lstlisting}[language=bash]
cd /opt/jsroot
\end{lstlisting}

\item Перейдите в основную ветвь репозитория:

\begin{lstlisting}[language=bash]
sudo git checkout master
\end{lstlisting}

\item Обновите исходные коды:

\begin{lstlisting}[language=bash]
sudo git pull origin master
\end{lstlisting}

\end{enumerate}

\section{Установка \MIDAS{} и \ROOTANA{}}

Данный раздел --- упрощённое изложение официальной инструкции по сборке \MIDAS{} \cite{MidasWikiInstall}. Оно не освещает все нюансы и возможные проблемы, с которыми можно столкнуться в процессе установки. В этом случае рекомендуем обратиться к официальной инструкции.

Если предполагается работа в многопользовательском режиме (стр.~\pageref{sec_multiuser}), то рекомендуется установить \MIDAS{} в каталог \DIRECTORY{/opt/midas}, в противном случае рекомендуемый каталог~--- \DIRECTORY{\~{}/packages/midas}. В дальнейших примерах предполагается первый вариант установки. Если же \MIDAS{} устанавливается в домашний каталог, то скорректируйте названия директорий соответствующим образом и  опустите команду \APP{sudo}.

\begin{enumerate}
\item Перейдите в каталог \DIRECTORY{/opt}:

\begin{lstlisting}[language=bash]
cd /opt
\end{lstlisting}

\item Скачайте репозитории с исходными кодами \MIDAS{} и библиотек:

\begin{lstlisting}[language=bash]
sudo git clone https://bitbucket.org/tmidas/midas
sudo git clone https://bitbucket.org/tmidas/mxml
sudo git clone https://bitbucket.org/tmidas/rootana
\end{lstlisting}

\item Перейдите в основную ветвь репозитория и запустите сборку \MIDAS{}:

\begin{lstlisting}[language=bash]
cd midas
sudo git checkout master
sudo git pull origin master
sudo make
\end{lstlisting}

\item Запустите сборку \ROOTANA{}:

\begin{lstlisting}[language=bash]
cd ../rootana
sudo git checkout master
sudo git pull origin master
sudo PATH=$ROOTSYS/bin:$PATH \
  LD_LIBRARY_PATH=$ROOTSYS/lib:$LD_LIBRARY_PATH make
\end{lstlisting}

\item Настройте переменные окружения. Для этого добавьте следующие строки в файл \FILE{\~{}/.profile}:

\begin{lstlisting}[language=bash]
export MIDASSYS=/opt/midas
export PATH=$MIDASSYS/linux/bin:$PATH
export ROOTANASYS=/opt/rootana
\end{lstlisting}

\item Настройте переменные окружения \ENV{MIDAS\_EXPTAB} и \ENV{MIDAS\_EXPT\_NAME}, необходимые для работы \MIDAS{}. Например, если эксперимент называется {\tt superexp}, то добавьте следующие строки в файл \FILE{\~{}/.profile}:

\begin{lstlisting}[language=bash]
export MIDAS_EXPTAB=$HOME/online/exptab
export MIDAS_EXPT_NAME=superexp
\end{lstlisting}

Перед началом работы с \MIDAS{} рекомендуется ознакомиться с описанием этих переменных и правилами их назначения в \cite{MidasWikiInstall}.

\end{enumerate}

Для того, чтобы изменения вступили в силу, завершите сеанс работы в ОС Linux и начните новый.

Если \MIDAS{} был установлен ранее и необходимо обновить его версию, то выполните следующие команды:

\begin{enumerate}
\item Перейдите в каталог \DIRECTORY{/opt/midas}:

\begin{lstlisting}[language=bash]
cd /opt/midas
\end{lstlisting}

\item Перейдите в основную ветвь репозитория:

\begin{lstlisting}[language=bash]
sudo git checkout master
\end{lstlisting}

\item Обновите исходные коды:

\begin{lstlisting}[language=bash]
sudo git pull origin master
\end{lstlisting}

\item Пересоберите приложения и библиотеки \MIDAS{}:

\begin{lstlisting}[language=bash]
sudo make
\end{lstlisting}

\end{enumerate}

Аналогичным образом производится обновление библиотеки \ROOTANA{}.

\section{Установка ПО для устройств CAEN}

\subsection{Установка библиотек CAEN}

Для связи с дигитайзерами компании CAEN необходимо скачать с официального сайта \cite{CaenHome} следующие компоненты:

\begin{itemize}

\item <<CAENVMELib Library>> -- библиотека для работы с VME устройствами.

\item <<CAENComm Library>> --- библиотека, реализующая связь с устройствами по оптоволокну.

\item <<CAENDigitizer Library>> --- библиотека по управлению дигитайзерами.

\end{itemize}

На странице каждого из компонентов необходимо выбрать раздел <<Downloads>> и скачать версию для Linux в формате TGZ. \NOTE{} для скачивания может потребоваться регистрация на сайте.

После скачивания приступите к установке:

\begin{enumerate}

\item Создайте в домашнем каталоге подкаталог \DIRECTORY{\~{}/opt/caen} и перейдите в него:

\begin{lstlisting}[language=bash]
cd
mkdir -p opt/caen
cd opt/caen
\end{lstlisting}

\item Скопируйте архивы с библиотеками CAEN в каталог \DIRECTORY{\~{}/opt/caen}.

\item Распакуйте архивы (здесь и далее номера версий в именах файлов и каталогов условные и могут отличаться):

\begin{lstlisting}[language=bash]
tar -xzf CAENVMELib-2.50.tgz
tar -xzf CAENComm-1.2-build20140211.tgz
tar -xzf CAENDigitizer_2.12.0_20181015.tgz
\end{lstlisting}

\item После распаковки в каталоге \DIRECTORY{\~{}/opt/caen} должны появиться следующие подкаталоги:

\begin{lstlisting}[language=bash]
CAENVMELib-2.50
CAENComm-1.2
CAENDigitizer_2.12.0
\end{lstlisting}

\item Перейдите в каталог \DIRECTORY{CAENVMELib-2.50/lib} и запустите установку:

\begin{lstlisting}[language=bash]
cd CAENVMELib-2.50/lib
sudo ./install_x64
\end{lstlisting}

Здесь и далее: если у вас 32-битная версия Linux, то для установки используйте команды без суфикса {\tt 64}:

\begin{lstlisting}[language=bash]
sudo ./install
\end{lstlisting}

\item Перейдите в каталог \DIRECTORY{CAENComm-1.2/lib} и запустите установку:

\begin{lstlisting}[language=bash]
cd ~/opt/caen
cd CAENComm-1.2/lib
sudo ./install_x64
\end{lstlisting}

\item Перейдите в каталог \DIRECTORY{CAENDigitizer\_2.12.0} и запустите установку:

\begin{lstlisting}[language=bash]
cd ~/opt/caen
cd CAENDigitizer_2.12.0
sudo ./install_64
\end{lstlisting}

\end{enumerate}

\subsection{Установка драйвера для CAEN~A3818}

Связь с дигитайзерами компании CAEN производится посредством A3818 --- карты-контроллера оптической сети CONET2.

Для работы с картой необходимо скачать с официального сайта \cite{CaenHome} и установить драйвер:

\begin{enumerate}

\item Создайте в домашнем каталоге подкаталог \DIRECTORY{\~{}/opt/caen} (если он не был создан ранее) и перейдите в него:

\begin{lstlisting}[language=bash]
cd
mkdir -p opt/caen
cd opt/caen
\end{lstlisting}

\item Найдите на сайте \cite{CaenHome} раздел, посвящённый карте-контроллеру A3818. Выберите подраздел <<Downloads>> и скачайте драйвер для Linux  в формате TGZ в каталог \DIRECTORY{\~{}/opt/caen}.

\item Распакуйте архив (здесь и далее номера версий в именах файлов и каталогов условные и могут отличаться):

\begin{lstlisting}[language=bash]
tar -xzf A3818Drv-1.6.2-build20181214.tgz
\end{lstlisting}

\item После распаковки в каталоге \DIRECTORY{\~{}/opt/caen} должен появиться подкаталог \DIRECTORY{A3818Drv-1.6.2}. Перейдите в него и запустите установку:

\begin{lstlisting}[language=bash]
make
sudo make install
\end{lstlisting}

\end{enumerate}

\section{Установка \GD{}}

\begin{enumerate}
\item Создайте в домашнем каталоге директорию \DIRECTORY{opt} (если она не была создана ранее) и перейдите в неё:

\begin{lstlisting}[language=bash]
cd
mkdir opt
cd opt
\end{lstlisting}

\item Скачайте репозиторий с исходными кодами \GD{}:

\begin{lstlisting}[language=bash]
git clone https://github.com/andrey-nakin/gneis-daq
\end{lstlisting}

\item Перейдите в подкаталог \DIRECTORY{gneis-daq}, создайте в нём подкаталог \DIRECTORY{obj} и перейдите в него:

\begin{lstlisting}[language=bash]
cd gneis-daq
mkdir obj
cd obj
\end{lstlisting}

\item Запустите сборку \GD{}:

\begin{lstlisting}[language=bash]
cmake ..
make
\end{lstlisting}

\item Установите выполнимые файлы \GD{} в системные каталоги{}:

\begin{lstlisting}[language=bash]
sudo make install
\end{lstlisting}

\end{enumerate}

Если \GD{} был установлен ранее и необходимо обновить его версию, то выполните следующие команды:

\begin{enumerate}
\item Перейдите в каталог \GD{}:

\begin{lstlisting}[language=bash]
cd ~/opt/gneis-daq
\end{lstlisting}

\item Обновите исходные коды:

\begin{lstlisting}[language=bash]
git pull origin master
\end{lstlisting}

\item Пересоберите и установите приложения:

\begin{lstlisting}[language=bash]
make
sudo make install
\end{lstlisting}

\end{enumerate}

%%%%%%%%%%%%%%%%%%%%%%%%%%%%%%%%%%%%%%%%%%%%%%%%%%%%%%%%%%%%%%%

\newcommand{\FE}{\APP{fe-v1720}}
\newcommand{\DEVICE}{CAEN~V1720}

\chapter{Фронтенд для \DEVICE{}}
\label{sec-fe-v1720}

%\documentclass[12pt, a4paper]{article}
%\usepackage[utf8]{inputenc}
%\usepackage[russian]{babel}
%\usepackage{gensymb}
%\usepackage{graphicx}
%\usepackage{indentfirst}
%\usepackage{url}

%\title{\APP{fe-v1720} --- MIDAS фронтенд-приложение для дигитайзера \DEVICE{}}
%\author{Накин~А.~В.}

%\begin{document}

%\maketitle

\section{Введение}

Приложение \FE{} является фронтендом \cite{MidasWikiFrontend} в системе сбора данных \MIDAS{} (стр.~\pageref{sec-midas-frontend}). Назначение приложения~--- управление дигитайзером \DEVICE{} и чтение wave-форм с него.

\section{Принцип работы}
\label{sec_basic}

\FE{} обеспечивает работу \DEVICE{} в следующем режиме:

\begin{itemize}
\item чтение wave-форм осуществляется с одного или нескольких каналов;
\item длина wave-форм одинакова для всех каналов;
\item запись wave-форм начинается по сигналу триггера одновременно по всем выбранным каналам и заканчивается по достижении нужной длины wave-форм;
\item с момента срабатывания триггера и до окончания записи wave-форм прочие сигналы триггера игнорируются;
\item параметры работы, такие как длина wave-форм, уровень порогового импульса и пр. задаются перед началом работы и хранятся в базе данных ODB \cite{MidasWikiODB}, которая хранит всю информацию об эксперименте в системе \MIDAS{}.
\end{itemize}

\subsection{Формирование триггерного сигнала}
\label{sec-v1720-trigger}

В качестве триггерного сигнала может выступать:

\begin{itemize}

\item Превышение порогового значения импульса по одному или нескольким выделенным каналам, называемым \TERM{триггерными}. В документации CAEN данный режим работы называется \TERM{Self-Trigger}  \cite{CaenUM3051ST}.  Триггерные импульсы могут иметь как положительную, так и отрицательную полярность.

\item Поступление внешнего сигнала на вход <<TRG-IN>> на передней панели \DEVICE{}. В документации CAEN данный режим работы называется \TERM{External Trigger} \cite{CaenUM3051ET}.

\end{itemize}

Возможно использование нескольких источников сигнала одновременно, в этом случае триггер формируется при наступлении любого из событий.

При срабатывании триггера от любого из источников подаётся сигнал на выход <<TRG-OUT>>, что позволяет синхронизировать внешние устройства.

Если используется несколько триггерных каналов, то один из них может быть помечен как \TERM{главный}. Главный триггерный канал используется анализаторами для построения временных гистограмм.

\subsection{Порядок опроса устройства}

Опрос устройства производится следующим образом:

\begin{enumerate}

\item 
\label{item-fe-v1720-poll-check}
\FE{} проверяет, имеются ли в памяти \DEVICE{} события, готовые для чтения.

\item Если событий нет, \FE{} выдерживает паузу, после чего переходит к первому пункту. По умолчанию пауза составляет 10~мс, это значение можно переопределить (см. раздел~\ref{fe-v1720-common-params} на стр.~\pageref{fe-v1720-common-params}).

\item Если в памяти \DEVICE{} имеются готовые события, \FE{} последовательно считывает их до тех пор, пока память не очистится.

\end{enumerate}

\subsection{Поддерживаемые прошивки}

\FE{} работает только с прошивкой <<Waveform Recording>>. Прошивка <<DPP-PSD>> не поддерживается. При попытке работы с дигитайзером, в котором прописана данная прошивка, в систему сообщений \MIDAS{} выдаётся сообщение об ошибке.

\section{Подготовка к работе}

Максимальный размер одного события, приходящего от \DEVICE{}, составляет чуть более 16~Мб, что превышает значение принятое по умолчанию в системе \MIDAS{} и равное 4194304~байтам.

Перед началом работы с \FE{} необходимо скорректировать это значение в следующем разделе базе данных ODB:

\medskip
{\tt /Experiment/MAX\_EVENT\_SIZE}
\medskip

Рекомендуемое значение парамера~--- 20971520 (20~Мб).

\section{Запуск и параметры командной строки}

Приложение \FE{} запускается из командной строки или веб-ин\-тер\-фейса \MIDAS{} \cite{MidasWikiMhttpd} и не имеет собственного графического пользовательского интерфейса. Управление режимами работы осуществляется при помощи параметров командной строки.

Если в эксперименте задействовано одно единственное устройство \DEVICE{}, то приложение \FE{} запускается без параметров. При этом в базе данных ODB в разделе \ODBNODE{Equipment} \cite{MidasWikiEquipment} создаётся запись с названием \ODBNODE{v1720}.

Если в эксперименте задействовано несколько устройств \DEVICE{}, то приложение \FE{} должно быть запущено несколько раз, по числу устройств, с параметром командной строки <<{\tt -i {\it x}}>>, где $x$~--- порядковый номер устройства начиная с нуля. При этом в базе данных ODB создаются записи с названиями \ODBNODE{v172000}, \ODBNODE{v172001} и т.~д.

Полный список поддерживаемых параметров командной строки приведён в \cite{MidasWikiFrontend}. 

\section{Настройка параметров работы в ODB}

Параметры работы \DEVICE{} задаются в следующем разделе базы данных ODB:

\medskip

{\tt /Equipment/v1720{\it xx}}

\medskip

\noindent где $xx$~--- двузначный номер устройства, отсутствующий в случае наличия единственного устройства данного типа.

\subsection{Стандартные параметры}
\label{fe-v1720-common-params}

Стандартные параметры работы \DEVICE{} задаются в следующем разделе базы данных ODB:

\medskip

{\tt /Equipment/v1720{\it xx}/Common}

\medskip

Назначение некоторых параметров:

\begin{itemize}

\item 

\ODBNODE{Period} --- размер паузы между опросами устройства в мс.

\end{itemize}

Полное описание параметров приведено в \cite{MidasWikiEquipment}.

\subsection{Специфичные параметры}
\label{sec-v1720-specific-params}

Специфичные параметры работы \DEVICE{} задаются в следующем разделе базы данных ODB:

\medskip

{\tt /Equipment/v1720{\it xx}/Settings}

\medskip

Назначение параметров:

\begin{itemize}

\item \ODBNODE{link\_num} --- номер интерфейсной карты A2818 или A3818, используемой для связи с \DEVICE{}. Номер начинается с нуля. При использовании единственной карты параметр равен 0.

\item \ODBNODE{conet\_node} --- номер устройства в оптоволоконной сети CONET. Номер начинается с нуля. При наличии единственного устройства в сети CONET параметр равен 0.

\item \ODBNODE{vme\_base\_addr} --- базовый адрес устройства на шине VME, если доступ к \DEVICE{} осуществляется посредством контроллера V1718 или V2718. Если контроллер не используется, данный парамер равен 0.

\item \ODBNODE{waveform\_length} --- длина wave-форм в сэмплах. Допустимые значения: от 0 до 1048576 ($2^{20})$. Рекомендуется использовать значения, являющиеся степенями~2, например 512, 1024 и т.~д. В этом случае внутренняя память \DEVICE{} используется оптимальным образом.

\item \ODBNODE{channel\_enabled} --- каналы, имеющие значение \ODBNODE{y}, участвуют в сборе данных. Для экономии вычислительных и дисковых ресурсов рекомендуется отключать каналы, не имеющие полезного сигнала на входе.

\item \ODBNODE{channel\_dc\_offset} --- сдвиг уровня аналогового сигнала (\TERM{DC Offset}) \cite{CaenUM3051AIS} по каждому из каналов. Диапазон значений: от 0 до 65535. Значение 0 соответствует диапазону $-2 \div 0$~В. Значение 32768 соответствует диапазону $-1 \div 1$~В. Значение 65535 соответствует диапазону $0 \div 2$~В. 

\item \ODBNODE{external\_trigger} --- если указано значение \ODBNODE{y}, то используется внешний триггерный сигнал, подключённый ко входу <<TRG-IN>>.

\item \ODBNODE{trigger\_channel} --- каналы, являющиеся триггерными (стр.~\pageref{sec_basic}).

\item \ODBNODE{trigger\_threshold} --- пороговые уровни триггерных импульсов для каждого из каналов. Допустимые значения: от 0 до 4095.

\item \ODBNODE{trigger\_raising\_polarity} --- типы триггерных импульсов для каждого из каналов. Если указано значение \ODBNODE{y}, триггер реагирует на положительные импульсы, в противном случае --- на отрицательные.

\item \ODBNODE{master\_trigger\_channel} --- номер главного триггерного канала (стр.~\pageref{sec-v1720-trigger}). Если указано отрицательное значение, главных триггерных каналов в данном эксперименте не определено.

\item 
\label{item-pre-trigger-length}

\ODBNODE{pre\_trigger\_length} --- количество сэмплов в wave-формах, предшествующих триггерному сигналу. Допустмые значения: от 0 до значения параметра \ODBNODE{waveform\_length}. В связи с особенностями внутреннего программного обеспечения \DEVICE{} \cite{CaenUM5961PostTrigger} фактическое количество сэмплов будет меньше указанного (на величину порядка 20-30 сэмплов), поэтому данный параметр должен задаваться с соответствующим <<запасом>>.

\item \ODBNODE{front\_panel\_io\_level} --- уровень электрических сигналов, подаваемых и снимаемых с передней панели \DEVICE{}, таких как <<TRG-IN>> или <<TRG-OUT>>. Допустимые значения: \ODBNODE{nim} или \ODBNODE{ttl}. На передней панели \DEVICE{} имеется индикация выбранного уровеня.

\end{itemize}

\section{Формат результирующих данных}
\label{sec_v1720_event}

Приложение \FE{} формирует отдельное событие (стр.~\pageref{sec-midas-event}) при каждом полезном срабывании триггера. Событие включает в себя следующие банки данных:

\begin{itemize}

\item \BANK{V200}~--- общая информация о событии;
\item \BANK{TRIG}~--- параметры триггера;
\item \BANK{CHDC}~--- значения DC offset для каналов дигитайзера на момент формирования события;
\item \BANK{WF00}~..~\BANK{WF07}~--- wave-формы с нулевого по седьмой каналы.

\end{itemize}

\subsection{Банк \BANK{V200}}
\label{sec_bank_info}

Данный банк содержит общую информацию о событии. Формат банка: 32-битные беззнаковые слова. Размер банка: 8 слов. Назначение слов приведено в таблице~\ref{tab-info-bank}.

\begin{table*}[h]
\centering
\begin{tabular}{ll}
\hline\hline
№ слова & Значение \\
\hline

0 & Board ID \\
1 & Битовая маска каналов \\
2 & Счётчик событий \\
3 & Trigger time stamp \cite{CaenUM3051TS}, младшие 32 бита \\
4 & Trigger time stamp, старшие 32 бита \\
5 & Номер устройства или {\tt 0xFFFFFFFF} \\
6 & Значение параметра \ODBNODE{pre\_trigger\_length} (стр.~\pageref{item-pre-trigger-length}) \\
7 & Зарезервировано, равно 0 \\

\hline\hline
\end{tabular}
\caption{Назначение слов в банке \BANK{V200}.}
\label{tab-info-bank}
\end{table*}

Битовая маска содержит информацию о каналах, участвующих в сборе данных. Если $i$-ый канал включён в сбор данных, то $i$-ый бит в маске установлен в единицу.

Если в системе используется одно единственное устройство \DEVICE{}, то слово №~5 имеет значение {\tt 0xFFFFFFFF}.

Наличие банка \BANK{V200} в событии служит индикатором того, что событие пришло от устройства \DEVICE{}.

\subsection{Банк \BANK{TRIG}}
\label{sec_bank_trig}

Данный банк содержит информацию о параметрах триггерных каналов. Формат банка: 16-битные беззнаковые слова. Размер банка: динамический, по 4 слова для каждого триггерного канала. Назначение слов приведено в таблице~\ref{tab-trig-bank}.

\begin{table*}[h]
\centering
\begin{tabular}{ll}
\hline\hline
№ слова & Значение \\
\hline

0 & Номер триггерного канала. \\
1 & Порог срабатывания триггера. \\
2 & Бит 0: \BITVALUE{1} при положительной полярности триггерного импульса, \\
 & иначе \BITVALUE{0}. \\
 & Бит 1: \BITVALUE{1} если триггерый канал является главным, \\
 & иначе \BITVALUE{0}. \\
 & Биты 15--2: зарезервировано, равно 0. \\
3 & Зарезервировано, равно 0. \\

\hline\hline
\end{tabular}
\caption{Назначение слов в банке \BANK{TRIG}.}
\label{tab-trig-bank}
\end{table*}

Банк не создаётся, если в качестве источника триггера используется только внешний сигнал (стр.~\pageref{sec-v1720-trigger}).

\subsection{Банк \BANK{CHDC}}

Банк содержит значения DC offset \cite{CaenUM3051AIS} по всем 8 каналам дигитайзера, включая каналы, незадействованные в сборе данных.

Формат банка: 16-битные беззнаковые слова. Размер банка: 8 слов. Слово №~0 содержит значение DC offset для канала №~0, слово №~1~--- для канала №~1 и так далее.

\subsection{Банки \BANK{WF0{\it i}}}

Банки данного типа  содержат данные wave-формы по одному из каналов. Название банка имеет вид \BANK{WF0{\it i}}, где $i$~--- номер канала начиная с нуля, например \BANK{WF00}, \BANK{WF01} и так далее.

Формат банка: 16-битные беззнаковые слова. Размер банка: произвольный, соответствует выбранной длине wave-форм. Каждое слово соответствует одному сэмплу wave-формы, при этом задействованы только младшие 12 битов, старшие 4 бита не используются и равны нулю.

Наличие банка в составе события зависит от того, участвует ли данный канал в сборе данных. Если канал $i$ включён, то $i$-ый бит в битовой маске банка \BANK{V200} (стр.~\pageref{sec_bank_info}) установлен в единицу, и банк \BANK{WF0{\it i}} входит в состав события.

%\begin{thebibliography}{99} 

%\bibitem{MidasWikiFrontend}  {\url{https://midas.triumf.ca/MidasWiki/index.php/Frontend_Application}} 

%\bibitem{MidasWiki}  {\url{https://midas.triumf.ca/MidasWiki/index.php/Main_Page}} 

%\bibitem{CaenUM3051ST}  {CAEN User Manual UM3051, раздел <<Self-Trigger>>, стр.~41} 

%\bibitem{MidasWikiODB}  {\url{https://midas.triumf.ca/MidasWiki/index.php/ODB_Access_and_Use}} 

%\bibitem{MidasWikiMhttpd}  {\url{https://midas.triumf.ca/MidasWiki/index.php/Mhttpd}} 

%\bibitem{MidasWikiEquipment}  {\url{https://midas.triumf.ca/MidasWiki/index.php//Equipment_ODB_tree}} 

%\bibitem{CaenUM3051AIS}  {CAEN User Manual UM3051, раздел <<Analog Input Stage>>, стр.~22} 

%\bibitem{CaenUM5961PostTrigger}  {CAEN User Manual UM5961, раздел <<Post Trigger>>, стр.~26} 

%\bibitem{MidasWikiEvent}  {\url{https://midas.triumf.ca/MidasWiki/index.php/Event_Structure}} 

%\bibitem{CaenUM3051TS}  {CAEN User Manual UM3051, раздел <<Technical Specifications>>, стр.~12} 

%\end{thebibliography}

%\end{document}



%%%%%%%%%%%%%%%%%%%%%%%%%%%%%%%%%%%%%%%%%%%%%%%%%%%%%%%%%%%%%%%

\renewcommand{\FE}{\APP{fe-v1724}}
\renewcommand{\DEVICE}{CAEN~V1724}

\chapter{Фронтенд для \DEVICE{}}

\section{Введение}

Приложение \FE{} является фронтендом для дигитайзера \DEVICE{} в системе сбора данных \MIDAS{} (стр.~\pageref{sec-midas-frontend}). Назначение приложения~--- управление дигитайзером и чтение wave-форм с него.

В целом, принцип работы и конфигурация данного приложения совпадают с фронтендом для CAEN~V1720 (стр.~\pageref{sec-fe-v1720}). Поэтому в данном разделе будут рассмотрены только отличия, характерные для \DEVICE{}.

\section{Настройка параметров работы в ODB}

Параметры работы \DEVICE{} задаются в следующем разделе базы данных ODB:

\medskip

{\tt /Equipment/v1724{\it xx}}

\medskip

\noindent где $xx$~--- двузначный номер устройства, отсутствующий в случае наличия единственного устройства данного типа.

\subsection{Специфичные параметры}

Специфичные параметры работы \DEVICE{} задаются в следующем разделе базы данных ODB:

\medskip

{\tt /Equipment/v1724{\it xx}/Settings}

\medskip

В целом, назначение параметров аналогично параметрам фронтенда для V1720 (стр.~\pageref{sec-v1720-specific-params}). Ниже в данном разделе приведено описание только отличающихся параметров:

\begin{itemize}

\item \ODBNODE{waveform\_length} --- длина wave-форм в сэмплах. Допустимые значения: от 0 до 524288 ($2^{19})$. Рекомендуется использовать значения, являющиеся степенями~2, например 512, 1024 и т.~д. В этом случае внутренняя память \DEVICE{} используется оптимальным образом.

\item \ODBNODE{channel\_dc\_offset} --- сдвиг уровня аналогового сигнала (\TERM{DC Offset}) \cite{CaenUM3248AIS} по каждому из каналов. Диапазон значений: от 0 до 65535. Значение 0 соответствует диапазону $-2.25 \div 0$~В. Значение 32768 соответствует диапазону $-1.125 \div 1.125$~В. Значение 65535 соответствует диапазону $0 \div 2.25$~В. 

\item \ODBNODE{trigger\_threshold} --- пороговый уровень триггерного импульса. Допустимые значения: от 0 до 16383.

\end{itemize}

\section{Формат результирующих данных}

Приложение \FE{} формирует отдельное событие (стр.~\pageref{sec-midas-event}) при каждом полезном срабывании триггера. Событие включает в себя следующие банки данных:

\begin{itemize}

\item \BANK{V240}~--- общая информация о событии;
\item \BANK{TRIG}~--- параметры триггера;
\item \BANK{CHDC}~--- значения DC offset для каналов дигитайзера на момент формирования события;
\item \BANK{WF00}~..~\BANK{WF07}~--- wave-формы с нулевого по седьмой каналы.

\end{itemize}

Формат банков аналогичен форматам банков для фронтенда \APP{fe-v1720} (стр.~\pageref{sec_v1720_event}). Отличие только в наличии банка \BANK{V240}, специфичного для \DEVICE{}.

\subsection{Банк \BANK{V240}}
\label{sec_v1724_bank_info}

Данный банк содержит общую информацию о событии. Формат банка: 32-битные беззнаковые слова. Размер банка: 8 слов. Назначение слов приведено в таблице~\ref{tab-info-bank}.

Наличие банка \BANK{V240} в событии служит индикатором того, что событие пришло от устройства \DEVICE{}.

Формат банка аналогичен формату банка \BANK{V200} для фронтенда \APP{fe-v1720} (стр.~\pageref{sec_bank_info}).

%%%%%%%%%%%%%%%%%%%%%%%%%%%%%%%%%%%%%%%%%%%%%%%%%%%%%%%%%%%%%%%

\appendix

\chapter{Алгоритм поиска вспышки в wave-форме}

Имеется wave-форма $S$ в виде упорядоченной последовательности сэмплов~$s_i$, где $i = 1..N$, N~--- количество сэмплов в wave-форме.

Получим новую wave-форму $P$ путём вычитания со сдвигом $d > 0$:

$$
p_j = s_{j+d} - s_j, j = 1..N-d,
$$

\noindent где $d$~--- параметр, выбираемый в соответствии с типичной длиной фронта вспышки, $d \ll N$.

Поскольку вспышки имеют некую среднюю продолжительность, то вводим параметр $l > d$, равный количеству сэмплов в типичной вспышке. 

Пусть $\bar{p}$~--- среднее значение $p_j$:

$$
\bar{p} = \frac{1}{N-d} \sum_{k=1}^{N-d} p_k,
$$

\noindent а $\sigma_p^2$~--- дисперсия:

$$
\sigma_p^2 = \frac{1}{N-d-1} \sum_{k=1}^{N-d} \left( p_k - \bar{p} \right) ^2.
$$

Стандартное отклонение $\sigma_p$ будем использовать как меру зашумлённости $S$.

Введём параметр $\kappa > 0$, определяющий величину порога срабатывания алгоритма.

Тогда будем считать, что в исходной wave-форме $S$ вспышка начинается с сэмпла с номером $i$, если имеет место следующее неравенство:

\begin{equation}
p_i \ge \kappa \cdot \sigma_p,
\end{equation}

\noindent если срабатывание по положительному фронту, или же:

\begin{equation}
p_i \le -\kappa \cdot \sigma_p,
\end{equation}
\noindent если срабатывание по отрицательному фронту.

Будучи обнаруженной, вспышка продолжается до сэмпла с номером $(i + l - 1)$ включительно. Далее, поиск вспышек продолжается с сэмпла с номером $(i + l)$.

Таким образом, алгоритм определяется набором параметров $(d, l, \kappa)$.

%%%%%%%%%%%%%%%%%%%%%%%%%%%%%%%%%%%%%%%%%%%%%%%%%%%%%%%%%%%%%%%

\printbibliography

\end{document}
