\documentclass[12pt, a4paper]{article}
\usepackage[utf8]{inputenc}
\usepackage[russian]{babel}
\usepackage{gensymb}
\usepackage{graphicx}
\usepackage{indentfirst}
\usepackage{url}

\newcommand{\BANK}[1]{{\tt #1}}
\newcommand{\APP}{{\tt fe-v1720}}
\newcommand{\DEVICE}{CAEN~V1720}
\newcommand{\TERM}[1]{{\it #1}}
\newcommand{\ODBNODE}[1]{<<{\tt #1}>>}

\title{\APP{} --- MIDAS фронтенд-приложение для дигитайзера \DEVICE{}}
\author{Накин~А.~В.}

\begin{document}

\maketitle

\section{Введение}

Приложение \APP{} является \TERM{фронтендом} (\TERM{Frontend Application}) \cite{MidasWikiFrontend} в системе сбора данных MIDAS \cite{MidasWiki}. Назначение программы~--- управление дигитайзером \DEVICE{} и чтение данных с него.

\section{Запуск и параметры командной строки}

Приложение \APP{} запускается из командной строки или веб-интерфейса MIDAS \cite{MidasWikiMhttpd} и не имеет собственного графического пользовательского интерфейса. Управление режимами работы осуществляется при помощи параметров командной строки.

Если в эксперименте задействовано одно единственное устройство \DEVICE{}, то приложение \APP{} запускается без параметров. При этом в базе данных ODB в ветке \ODBNODE{Equipment} \cite{MidasWikiEquipment} создаётся запись с названием \ODBNODE{v1720}.

Если в эксперименте задействовано несколько устройств \DEVICE{}, то приложение \APP{} должно быть запущено несколько раз, по числу устройств, с параметром командной строки <<{\tt -i {\it x}}>>, где $x$~--- порядковый номер устройства начиная с нуля. При этом в базе данных ODB создаются записи с названиями \ODBNODE{v172000}, \ODBNODE{v172001} и т.~д.

Полный список поддерживаемых параметров командной строки приведён в \cite{MidasWikiFrontend}. 

\section{Формат события}

В системе MIDAS единицей хранения информации является \TERM{событие} (\TERM{Event}), которое в свою очередь состоит из одного или нескольких \TERM{банков} (\TERM{Bank}) \cite{MidasWikiEvent}. Каждый банк имеет 4-х буквенное название, по которому определяется его тип. Событие может включать в себя банки разного типа.

Событие, сформированное приложением \APP{}, включает в себя следующие банки данных:

\begin{itemize}

\item \BANK{V200} ~--- общая информация о событии;
\item \BANK{CHDC} ~--- значения DC offset для каналов дигитайзера на момент формирования события;
\item \BANK{WF00}~..~\BANK{WF07} ~--- wave-формы для каналов с нулевого по седьмой.

\end{itemize}

\subsection{Банк \BANK{V200}}

Данный банк содержит общую информацию о событии. Формат банка: 32-битные беззнаковые слова. Размер банка: 5 слов. Назначение слов приведено в таблице~\ref{tab-info-bank}.

\begin{table}[h]
\centering
\begin{tabular}{rl}
\hline\hline
№ слова & Значение \\
\hline

0 & Board ID \\
1 & Битовая маска каналов \\
2 & Счётчик событий \\
3 & Timestamp, младшие 32 бита \\
4 & Timestamp, старшие 32 бита \\

\hline\hline
\end{tabular}
\caption{Назначение слов в банке \BANK{V200}.}
\label{tab-info-bank}
\end{table}

Битовая маска содержит информацию о каналах, участвующих в сборе данных. Если $i$-ый канал включён в сбор данных, то $i$-ый бит в маске установлен в единицу.

Наличие банка \BANK{V200} в событии служит индикатором того, что событие пришло от устройства \DEVICE{}.

\subsection{Банк \BANK{CHDC}}

Банк содержит значения настройки DC offset по всех 8 каналам дигитайзера, включая каналы, незадействованные в сборе данных.

Формат банка: 16-битные беззнаковые слова. Размер банка: 8 слов. Слово №~0 содержит значение DC offset для канала №~0, слово №~1 содержит значение DC offset для канала №~1 и так далее.

\subsection{Банки \BANK{WF0{\it i}}}

Банки имеют название типа \BANK{WF0{\it i}}, где $i$~--- номер канала начиная с нуля, например \BANK{WF00}, \BANK{WF01} и так далее.

Наличие банка в составе события зависит от того, участвует ли данный канал в сборе данных. Если канал $i$ включён, то $i$-ый бит в битовой маске банка \BANK{V200} установлен в единицу, и банк \BANK{WF0{\it i}} входит в состав события.

Формат банка: 16-битные беззнаковые слова. Размер банка: произвольный, соответствует выбранному размеру wave-формы. Каждое слово соответствует одному сэмплу wave-формы.

\begin{thebibliography}{99} 

\bibitem{MidasWikiFrontend}  {\url{https://midas.triumf.ca/MidasWiki/index.php/Frontend_Application}} 

\bibitem{MidasWiki}  {\url{https://midas.triumf.ca/MidasWiki/index.php/Main_Page}} 

\bibitem{MidasWikiMhttpd}  {\url{https://midas.triumf.ca/MidasWiki/index.php/Mhttpd}} 

\bibitem{MidasWikiEquipment}  {\url{https://midas.triumf.ca/MidasWiki/index.php//Equipment_ODB_tree}} 

\bibitem{MidasWikiEvent}  {\url{https://midas.triumf.ca/MidasWiki/index.php/Event_Structure}} 

\end{thebibliography}

\end{document}
